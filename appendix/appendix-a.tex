\chapter{Feedback plots}
\label{appendix:feedback-plots}
%\appendix




% \section{Feedback task plate-slide-v2}
% \label{section:Feedback plots}


The following feedback results show the behaviour of the simulated oracles that were implemented to run the simulated experiments for the three tasks plate-slide-v2, drawer-open-v2 and button-press-topdown-v2 explained in Section \ref{section:Meta-World Benchmark}. 



% Figure \ref{fig:results_plate_slide_buffer_e} shows the average success per episode for the task plate-slide-v2 for different values of $e$ and buffer size when the positions are relative. There are three main conclusions that can be extracted. First, BD-COACH is overall much more robust to changes in both $e$ and buffer size $K$ and it is able to reach 100\% success around minute 6. Regarding D-COACH, $e$ is the parameter that has more influence being its worst performance when $e=0.01$. And regarding the buffer size $K$, the biggest buffer is the most detrimental and the smallest results in the best performance at the end of the training.
% This was expected as in D-COACH the buffer contains information gathered by all the previous old versions of the policy that may not be useful for updating the current version of the policy.


% Same buffer - feedback
 \begin{figure}[H]
  \centering
  \captionsetup[subfigure]{oneside,margin={0.85cm,0cm}}
  \subfloat[(a) plate-slide-v2]{\adjustbox{trim=0cm 0cm 0cm 0cm}{%
  \includesvg[width=.33\textwidth]{figures/hockey-same-buffer-feedback.svg}}\label{fig:hockey-same-buffer-feedback}}
   \hfill
  \subfloat[(b) drawer-open-v2]{\adjustbox{trim=0cm 0cm 0cm 0cm}{%
  \includesvg[width=.33\textwidth]{figures/drawer-same-buffer-feedback.svg}}\label{fig:drawer-same-buffer-feedback}}
  \hfill
  \subfloat[(c) button-press-topdown-v2]{\adjustbox{trim=0cm 0cm 0cm 0cm}{%
  \includesvg[width=.33\textwidth]{figures/button-same-buffer-feedback.svg}}\label{fig:button-same-buffer-feedback}}
  \caption{Results of the percentage of time steps per episode that the simulated teacher, $P_h: \alpha = 0.9; \tau =  0.000015$,  provides feedback for a fixed buffer size $K$ and different values of $e$. These feedback plots correspond to the success results of Figure \ref{fig:same-buffer}.}
 
  \label{fig:same-buffer-feedback}
\end{figure}



% Same e - feedback
 \begin{figure}[H]
  \centering
\captionsetup[subfigure]{oneside,margin={0.85cm,0cm}}
  \subfloat[(a) plate-slide-v2]{\adjustbox{trim=0cm 0cm 0cm 0cm}{%
  \includesvg[width=.33\textwidth]{figures/hockey-same-e-feedback.svg}}\label{fig:hockey-same-e-feedback}}
   \hfill
  \subfloat[(b) drawer-open-v2]{\adjustbox{trim=0cm 0cm 0cm 0cm}{%
  \includesvg[width=.33\textwidth]{figures/drawer-same-e-feedback.svg}}\label{fig:drawer-same-e-feedback}}
  \hfill
  \subfloat[(c) button-press-topdown-v2]{\adjustbox{trim=0cm 0cm 0cm 0cm}{%
  \includesvg[width=.33\textwidth]{figures/button-same-e-feedback.svg}}\label{fig:button-same-e-feedback}}
  \caption{Results of the percentage of time steps per episode that the simulated teacher, $P_h: \alpha = 0.9; \tau =  0.000015$,  provides feedback for a fixed error magnitude $e$ and different values of the buffer size $K$. The observations for all the conditions shown are relative positions. These feedback plots correspond to the success results of Figure \ref{fig:same-e}.}
  \label{fig:same-e-feedback}
\end{figure}


% % Best - feedback
%  \begin{figure}[H]
%   \centering
%     \captionsetup[subfigure]{oneside,margin={0.85cm,0cm}}
%   \subfloat[(a) plate-slide-v2]{\adjustbox{trim=0cm 0cm 0cm 0cm}{%
%   \includesvg[width=.33\textwidth]{figures/hockey-best-feedback.svg}}\label{fig:hockey-best-feedback}}
%   \hfill
%   \subfloat[(b) drawer-open-v2]{\adjustbox{trim=0cm 0cm 0cm 0cm}{%
%   \includesvg[width=.33\textwidth]{figures/drawer-best-feedback.svg}}\label{fig:drawer-best-feedback}}
%   \hfill
%   \subfloat[(c) button-press-topdown-v2]{\adjustbox{trim=0cm 0cm 0cm 0cm}{%
%   \includesvg[width=.33\textwidth]{figures/button-best-feedback.svg}}\label{fig:button-best-feedback}}
%   \caption{Results of the percentage of time steps per episode that the simulated teacher, $P_h: \alpha = 0.9; \tau =  0.000015$, when comparing the best performance of each method using relative positions, against their performances when using absolute positions for the
% same conditions of $K$ and $e$. These feedback plots correspond to the success results of Figure \ref{fig:best}.}
%   \label{fig:best-feedback}
% \end{figure}



% Best - feedback
 \begin{figure}[H]
  \centering
    \captionsetup[subfigure]{oneside,margin={0.85cm,0cm}}
  \subfloat[(a) plate-slide-v2]{\adjustbox{trim=0cm 0cm 0cm 0cm}{%
  \includesvg[width=.33\textwidth]{figures/hockey-best-feedback-2.svg}}\label{fig:hockey-best-feedback}}
   \hfill
  \subfloat[(b) drawer-open-v2]{\adjustbox{trim=0cm 0cm 0cm 0cm}{%
  \includesvg[width=.33\textwidth]{figures/drawer-best-feedback-2.svg}}\label{fig:drawer-best-feedback}}
  \hfill
  \subfloat[(c) button-press-topdown-v2]{\adjustbox{trim=0cm 0cm 0cm 0cm}{%
  \includesvg[width=.33\textwidth]{figures/button-best-feedback-2.svg}}\label{fig:button-best-feedback}}
  \caption{Results of the percentage of time steps per episode that the simulated teacher, $P_h: \alpha = 0.9; \tau =  0.000015$, when comparing the best performance of each method using relative positions, against their performances when using absolute positions for the
same conditions of $K$ and $e$. These feedback plots correspond to the success results of Figure \ref{fig:best}.}
  \label{fig:best-feedback}
\end{figure}
