\chapter{Feedback plots}
\label{appendix:feedback-plots}
%\appendix




\section{Feedback task plate-slide-v2}
\label{section:Feedback plots}




Figure \ref{fig:results_plate_slide_buffer_e} shows the average success per episode for the task plate-slide-v2 for different values of $e$ and buffer size when the positions are relative. There are three main conclusions that can be extracted. First, BD-COACH is overall much more robust to changes in both $e$ and buffer size $K$ and it is able to reach 100\% success around minute 6. Regarding D-COACH, $e$ is the parameter that has more influence being its worst performance when $e=0.01$. And regarding the buffer size $K$, the biggest buffer is the most detrimental and the smallest results in the best performance at the end of the training.
This was expected as in D-COACH the buffer contains information gathered by all the previous old versions of the policy that may not be useful for updating the current version of the policy.

% Same buffer
 \begin{figure}[H]
  \centering
  \subfloat{\adjustbox{trim=0cm 0cm 1cm 0cm}{%
  \includesvg[width=.55\textwidth]{figures/plate-slide-v2-same-buffer-feedback.svg}}\label{fig:plate_slide_same_buffer}}
   \hfill
  \subfloat{\adjustbox{trim=0cm 0cm 1cm 0cm}{%
  \includesvg[width=.55\textwidth]{figures/plate-slide-v2-same-e-feedback.svg}}\label{fig:plate_slide_same_e}}
  \caption{plate-slide-v2 results using a simulated teacher $P_h: \alpha = 0.9; \tau =  0.000015$. On the left, a comparison of the influence of the $e$ parameter on the baseline method D-COACH and on BD-COACH. On the right, the percentage of time steps per episode that the teacher  provides feedback.}
  \label{fig:results_plate_slide_buffer_e}
\end{figure}



% best
 \begin{figure}[H]
  \centering
  \subfloat{\adjustbox{trim=0cm 0cm 1cm 0cm}{%
  \includesvg[width=.55\textwidth]{figures/plate-slide-v2-best-feedback.svg}}\label{fig:plate-slide-best}}

  \caption{plate-slide-v2 results using a simulated teacher $P_h: \alpha = 0.9; \tau =  0.000015$. On the left, a comparison of the best performance of each method when using relative positions, against their performances when using absolute positions with the same conditions of $K$ and $e$. On the right, the percentage of time steps per episode that the teacher  provides feedback.}
  \label{fig:results-plate-slide-best}
\end{figure}











      
From the previous analysis, the best combinations of buffer size $K$ and $e$ for both methods when positions are relative are $K=30000$ and $e=0.1$ for BD-COACH and $K=15000$ and $e=1$ for D-COACH. The experiments are repeated with the same values but this time considering absolute positions instead of relative ones. Figure \ref{fig:results-plate-slide-best} shows this comparison where BD-COACH, it is able to reach similar levels of performance as in the case of relative positions whereas the performance of D-COACH decreases a 20\%.


\subsection{Performance of task drawer-open-v2}
\label{subsection:Performance of task drawer_open_v2}




Similar to previous task plate-slide-v2, Figure \ref{fig:results_drawer-open-same-buffer} shows a comparison of the effect of different $e$ values and Figure \ref{fig:results_drawer-open-same-e} compares different buffer sizes $K$. For the task drawer-open-v2, again, BD-COACH behaves more robust than D-COACH for the different conditions, being able to always reach performances close to 100 \%. On the other hand, D-COACH performs better in general than in the task plate-slide-v2 task but again, it can be observed that when the value of $e$ is small its performance decreases.



 \begin{figure}[H]
  \centering
  \subfloat{\adjustbox{trim=0cm 0cm 1cm 0cm}{%
  \includesvg[width=.55\textwidth]{figures/drawer-open-v2-same-buffer-feedback.svg}}\label{fig:drawer-open-same-buffer}}
   \hfill
  \subfloat{\adjustbox{trim=0cm 0cm 1cm 0cm}{%
  \includesvg[width=.55\textwidth]{figures/drawer-open-v2-same-e-feedback.svg}}\label{fig:drawer-open-same-buffer-feedback}}
  \caption{drawer-open-v2 results using a simulated teacher $P_h: \alpha = 0.9; \tau =  0.000015$. On the left, a comparison of the influence of the $e$ parameter on the baseline method D-COACH and on BD-COACH. On the right, the percentage of time steps per episode that the teacher  provides feedback.}
  \label{fig:results_drawer-open-same-buffer}
\end{figure}





 \begin{figure}[H]
  \centering
  \subfloat{\adjustbox{trim=0cm 0cm 1cm 0cm}{%
  \includesvg[width=.55\textwidth]{figures/drawer-open-v2-best-feedback.svg}}\label{fig:drawer-open-best}}

  \caption{drawer-open-v2 results using a simulated teacher $P_h: \alpha = 0.9; \tau =  0.000015$. On the left, a comparison of the best performance of each method when using relative positions, against their performances when using absolute positions with the same conditions of $K$ and $e$. On the right, the percentage of time steps per episode that the teacher  provides feedback.}
  \label{fig:results_drawer-open-best}
\end{figure}





From the previous analysis, the best combinations of buffer size $K$ and $e$ for both methods when positions are relative are $K=30000$ and $e=0.1$ for BD-COACH and $K=3000$ and $e=0.1$ for D-COACH. The experiments are repeated with the same values but this time considering absolute positions instead of relative ones. Figure \ref{fig:results_drawer-open-best}  shows this comparison where BD-COACH is able to reach similar levels of performance as in the case of relative positions. In the case of D-COACH, its performance soon reaches an average success of 70\% which increases just at the end of the training. Still, for D-COACH the difference in success between relative and absolute positions is remarkable.


\subsection{Performance of task button-press-topdown-v2}
\label{subsection:Performance of task button-press-topdown-v2}


% BUTTON


 \begin{figure}[H]
  \centering
  \subfloat{\adjustbox{trim=0cm 0cm 1cm 0cm}{%
  \includesvg[width=.55\textwidth]{figures/button-press-topdown-v2-same-buffer-feedback.svg}}\label{fig:button-press-topdown-v2-same-buffer}}
   \hfill
  \subfloat{\adjustbox{trim=0cm 0cm 1cm 0cm}{%
  \includesvg[width=.55\textwidth]{figures/button-press-topdown-v2-same-e-feedback.svg}}\label{fig:button-press-topdown-v2-same-buffer-feedback}}
  \caption{button-press-topdown-v2 results using a simulated teacher $P_h: \alpha = 0.9; \tau =  0.000015$. On the left, a comparison of the influence of the $e$ parameter on the baseline method D-COACH and on BD-COACH. On the right, the percentage of time steps per episode that the teacher provides feedback.}
  \label{fig:results-button-press-topdown-v2-same-buffer}
\end{figure}




 \begin{figure}[H]
  \centering
  \subfloat{\adjustbox{trim=0cm 0cm 1cm 0cm}{%
  \includesvg[width=.55\textwidth]{figures/button-press-topdown-v2-best-feedback.svg}}\label{fig:button-press-topdown-v2-best}}
   \caption{button-press-topdown-v2 results using a simulated teacher $P_h: \alpha = 0.9; \tau =  0.000015$. On the left, a comparison of the best performance of each method when using relative positions, against their performances when using absolute positions with the same conditions of $K$ and $e$. On the right, the percentage of time steps per episode that the teacher  provides feedback.}
  \label{fig:resultsbutton-press-topdown-v2-best}
\end{figure}